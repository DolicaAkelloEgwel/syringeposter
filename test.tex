\documentclass[a0paper,landscape]{baposter}

\usepackage{relsize}		% For \smaller
\usepackage{epstopdf}	% Included EPS files automatically converted to PDF to include with pdflatex
\usepackage{float}
\usepackage{wrapfig}
\usepackage{minted}
\usepackage[none]{hyphenat}

%%% Global Settings %%%%%%%%%%%%%%%%%%%%%%%%%%%%%%%%%%%%%%%%%%%%%%%%%%%%%%%%%%%

\graphicspath{{pix/}}	% Root directory of the pictures 
\tracingstats=2			% Enabled LaTeX logging with conditionals

%%% Color Definitions %%%%%%%%%%%%%%%%%%%%%%%%%%%%%%%%%%%%%%%%%%%%%%%%%%%%%%%%%

\definecolor{bordercol}{RGB}{68,68,68}
\definecolor{headercol1}{RGB}{254,210,11}
\definecolor{headercol2}{RGB}{242,121,0}
\definecolor{headerfontcol}{RGB}{0,0,0}
\definecolor{boxcolor}{RGB}{255,255,230}

%%%%%%%%%%%%%%%%%%%%%%%%%%%%%%%%%%%%%%%%%%%%%%%%%%%%%%%%%%%%%%%%%%%%%%%%%%%%%%%%
%%% Utility functions %%%%%%%%%%%%%%%%%%%%%%%%%%%%%%%%%%%%%%%%%%%%%%%%%%%%%%%%%%

%%% Save space in lists. Use this after the opening of the list %%%%%%%%%%%%%%%%
\newcommand{\compresslist}{
	\setlength{\itemsep}{1pt}
	\setlength{\parskip}{0pt}
	\setlength{\parsep}{0pt}
}

%%%%%%%%%%%%%%%%%%%%%%%%%%%%%%%%%%%%%%%%%%%%%%%%%%%%%%%%%%%%%%%%%%%%%%%%%%%%%%%
%%% Document Start %%%%%%%%%%%%%%%%%%%%%%%%%%%%%%%%%%%%%%%%%%%%%%%%%%%%%%%%%%%%
%%%%%%%%%%%%%%%%%%%%%%%%%%%%%%%%%%%%%%%%%%%%%%%%%%%%%%%%%%%%%%%%%%%%%%%%%%%%%%%

\begin{document}
\typeout{Poster rendering started}

%%% Setting Background Image %%%%%%%%%%%%%%%%%%%%%%%%%%%%%%%%%%%%%%%%%%%%%%%%%%
\background{}

%%% General Poster Settings %%%%%%%%%%%%%%%%%%%%%%%%%%%%%%%%%%%%%%%%%%%%%%%%%%%
%%%%%% Eye Catcher, Title, Authors and University Images %%%%%%%%%%%%%%%%%%%%%%
\begin{poster}{
	grid=false,
	% Option is left on true though the eyecatcher is not used. The reason is
	% that we have a bit nicer looking title and author formatting in the headercol
	% this way
	eyecatcher=true, 
	borderColor=black,
	headerColorOne=blue,
	headerColorTwo=blue,
	headerFontColor=white,
	% Only simple background color used, no shading, so boxColorTwo isn't necessary
	boxColorOne=boxcolor,
	headershape=rectangle,
	headerborder=closed,
	headerfont=\Large\sf\bf,
	textborder=rectangle,
	background=none,
	headerborder=open,
    boxshade=plain
}
%%% Title %%%%%%%%%%%%%%%%%%%%%%%%%%%%%%%%%%%%%%%%%%%%%%%%%%%%%%%%%%%%%%%%%%%%%
{\includegraphics[height=4em]{images/diamondlogo}} 
{\bf Syringe Pump for Electrochemical Cell \vspace{0.5em}} % Poster title
{\textbf{Dolica Akello-Egwel}, Gary Yendell \hspace{12pt} 2018 Cohort / Mantid Team} % Author names and institution
{\includegraphics[height=4em]{images/stfclogo}} 

%%%% Column 1 %%%%%%%%%%%%%%%%%%%%%%%%%%%%%%%%%%%%%%%%%%%%%%%%%%%%%%%%%%%%%%%%%%
%----------------------------------------------------------------------------------------
%	INTRODUCTION
%----------------------------------------------------------------------------------------
\begin{posterbox}[name=introduction,column=0]{Introduction}
The existing streamDevice-based EPICS driver for the Hamilton Microlab 500/600 Syringe Pump was not
especially suited to sending "complex" commands that contain two or more instructions.
The driver also lacked the ability to reliably send simultaneous commands to both syringes, and to
cycle sets of simultaneous commands indefinetly. The aim of this project was to create a user-friendly
and extensible Python API that allows users to easily create continuous flows through an electrochemical
cell.
\end{posterbox}

\begin{posterbox}[name=objectives,column=0,row=0,below=introduction]{Objectives}
 \begin{itemize}
    \item Make a Python library for controlling the Hamilton Microlab Syringe Pump
    \item Add EDM screens
    \item Add a cycling ability to allow
    \item Interface with EPICS.
\end{itemize}
\end{posterbox}

\begin{posterbox}[name=implementaiton,span=2,column=1]{Implementation}
\begin{minted}{python}
def _cycle_commands(
    self, commands: List[Command], address: str,
):
    """Cycles through lists of commands. Stops when the `stop_cycle` method is
    called.

    Args:
        command_lists (List[Command]): The list of commands.
        address (str): The target address for the commands.
    """
    # Avoid getting lots of log messages every time `send_receive` is called
    logging.getLogger(self._comms.__class__.__name__).setLevel(logging.ERROR)
    while not self._stop_requested:
        for command in commands:
            self.send_command(command, address)
            while (
                self._comms.send_receive(address + INSTRUMENT_DONE)[1]
                != Y_INFO_RESPONSE
            ):
                if self._stop_requested:
                    self._reset_cycle(address)
                    return
                sleep(0.1)
    self._reset_cycle(address)
\end{minted}
\end{posterbox}

\end{poster}
\end{document}
